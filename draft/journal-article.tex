\documentclass[USenglish]{article}	
% for 2-column layout use \documentclass[USenglish,twocolumn]{article}

\usepackage[utf8]{inputenc}				%(only for the pdftex engine)
%\RequirePackage[no-math]{fontspec}[2017/03/31]%(only for the luatex or the xetex engine)
\usepackage[big,online]{dgruyter}	%values: small,big | online,print,work
\usepackage{lmodern} 
\usepackage{microtype}
\usepackage[numbers,square,sort&compress]{natbib}
\usepackage{bm}
\usepackage{mathtools}

% New theorem-like environments will be introduced by using the commands \theoremstyle and \newtheorem.
% Please note that the environments proof and definition are already defined within dgryuter.sty.
\theoremstyle{dgthm}
\newtheorem{theorem}{Theorem}
\newtheorem{corollary}{Corollary}
\newtheorem{proposition}{Proposition}
\newtheorem{lemma}{Lemma}
\newtheorem{assertion}{Assertion}
\newtheorem{result}{Result}
\newtheorem{conclusion}{Conclusion}

\theoremstyle{dgdef}
\newtheorem{definition}{Definition}
\newtheorem{example}{Example}
\newtheorem{remark}{Remark}

\let\vec\bm
\newcommand\norm[1]{\lVert#1\rVert}
\newcommand\abs[1]{\lvert#1\rvert}

\begin{document}

	
%%%--------------------------------------------%%%
	\articletype{Research Article}
	%\received{Month	DD, YYYY}
	%\revised{Month	DD, YYYY}
 % \accepted{Month	DD, YYYY}
 % \journalname{De~Gruyter~Journal}
  %\journalyear{YYYY}
  %\journalvolume{XX}
  %\journalissue{X}
  \startpage{1}
  %\aop
  %\DOI{10.1515/sample-YYYY-XXXX}
%%%--------------------------------------------%%%

\title{On sufficient conditions for the well-posedness of mixed finite element methods for a class of eigenvalue problems}
\runningtitle{On sufficient conditions for mixed finite element eigenvalue problems}
%\subtitle{Insert subtitle if needed}

%\ use * to mark the author as the corresponding author
\author[1]{Fleurianne Bertrand}
\author[2]{Daniele Boffi} 
\author*[3]{Umberto Zerbinati}
\runningauthor{F.~Bertrand et al.}
\affil[1]{\protect\raggedright Chemnitz University of Technology, Chemnitz, Germany, e-mail: fleurianne.bertrand@mathematik.tu-chemnitz.de}
\affil[2]{\protect\raggedright King Abdullah University of Science and Technology, Thuwal, Saudi Arabia, e-mail: daniele.boffi@kaust.edu.sa}
\affil[3]{\protect\raggedright University of Oxford, Oxford, United Kingdom, e-mail: zerbinati@maths.ox.ac.uk}
	
%\communicated{...}
%\dedication{...}
	
\abstract{}

\keywords{}

\maketitle
	
	
\section{Introduction}
Eigenvalue problems play a fundamental role in engineering. In particular, in structural mechanics, the discretisation and solution of eigenvalue problems are at the heart of \textit{modal analysis}, i.e. the characterization of the response of an elastic body in terms of its natural modes of vibration.
Let us consider a body $\Omega\subset \mathbb{R}^d$, with linearly elastic Cauchy stress tensor, i.e $\sigma = \lambda (\nabla\cdot \vec{u})\mathbb{I}+2\mu\varepsilon(\vec{u})$, where $\vec{u}:\Omega\to \mathbb{R}^d$ reppresents the displacement of the elastic body and $\varepsilon(\vec{u})$ is the symmetric gradient of $\vec{u}$.
Under these hypotheses, the balance law for the linear momentum can be expressed in weak form as 
\begin{equation}
  \label{eq:elasticityPrimalSource}
  2\mu \int_{\Omega} \varepsilon(\vec{u}):\varepsilon(\vec{v})\, d\vec{x} + \lambda \int_{\Omega} (\nabla \cdot \vec{u})(\nabla\cdot \vec{v})\, d\vec{x} = \int_{\Omega}\vec{f}\cdot \vec{v} \,d\vec{x} \qquad \forall \vec{v}\in \vec{V},
\end{equation}
where $\vec{f}:\Omega\to \mathbb{R}^d$ is the bulk force actioning on the body and $\vec{V}$ is usual Sobolev space $\vec{H}^1_0(\Omega)$. The \textit{modal analysis} corresponding to the previous source problem can be formulated as the following eigenvalue problem:
\begin{equation}
  \label{eq:elasticityPrimalEig}
  2\mu \int_{\Omega} \varepsilon(\vec{u}):\varepsilon(\vec{v})\, d\vec{x} + \lambda \int_{\Omega} (\nabla \cdot \vec{u})(\nabla\cdot \vec{v})\, d\vec{x} = \omega \int_{\Omega}\vec{u}\cdot \vec{v} \,d\vec{x} \qquad \forall \vec{v} \in \vec{V}.
\end{equation}
Introducing the mechanical pressure $p:\Omega\to \mathbb{R}$ defined as the trace of the stress tensor, i.e. $p = 2(\lambda+\mu) \nabla\cdot \vec{u}$, the previous eigenvalue problem can be recast as the following eigenvalue problem:
\begin{equation}
  \label{eq:elasticityPressureEig}
    2\mu \int_\Omega \widehat\varepsilon(\vec{u}):\widehat \varepsilon(\vec{v})\, d\vec{x}+\int_\Omega p (\nabla \cdot \vec{v})\, d\vec{x} = \omega \int_\Omega \vec{u}\cdot \vec{v}\, d\vec{x} \qquad \vec{v}\in \vec{V},
\end{equation}
where $\widehat\varepsilon(\vec{u}) = \varepsilon(\vec{u})-\frac{1}{d}(\nabla\cdot \vec{u})\mathbb{I}$ is the deviatoric part of the symmetric gradient of $\vec{u}$.

It is a well-known fact that for nearly incompressible elasticity, i.e. $\lambda\to \infty$, finite element simulation can experience a degraded order of convergence or even converge to the wrong solution \cite{AinsworthParker}. Such phenomena are broadly referred to as \textbf{locking}. Over the years many different approaches have been proposed to overcome locking.
For instance, the use of high-order finite elements or of specific mesh topology is known to mitigate or even cure in certain cases the locking phenomenon, see \cite{AinsworthParker, BabuskaSuri, BBF} and reference therein.

Another widely adopted approach is to introduce a Lagrange multiplier to enforce the divergence-free condition, which is equivalent to adding to \eqref{eq:elasticityPrimalEig} the constraint
\begin{equation}
  \label{eq:lagrangeMultiplier}
  \int_\Omega (\nabla\cdot \vec{u})q = 0 \qquad \forall  q\in Q, \text{ with } Q=L^2(\Omega).
\end{equation}
Hence, in the limit $\lambda\to \infty $ the eigenvalue problem \eqref{eq:elasticityPressureEig} becomes the Stokes eigenvalue problem:
\begin{equation}
  \label{eq:stokes}
  \begin{aligned}
    2\mu \int_\Omega \varepsilon(\vec{u}):\varepsilon(\vec{v})\, d\vec{x}-\int_\Omega \nabla p \cdot \vec{v}\, d\vec{x} &= \omega \int_\Omega \vec{u}\cdot \vec{v}\, d\vec{x} \qquad &\vec{v}\in \vec{V},\\
    -\int_\Omega (\nabla\cdot \vec{u})q &= 0 \qquad&q \in Q,
  \end{aligned}
\end{equation}
where the pressure $p$ has now the meaning of being a Lagrange multiplier enforcing the divergence-free condition, rather than the mechanical pressure.
When considering the previous eigenvalue problem one can overcome the locking phenomenon if two conditions known as strong and weak approximability conditions are verified, a brief overview of such conditions can be found in section 2 and full detail can be found in \cite{BBG, BBF}.

In this paper, we investigate certain sufficient conditions that guarantee the well-posedness and optimal convergence rate of the mixed finite element method for the class of eigenvalue problems of the type of \eqref{eq:stokes}, also known as $(f\!-\!0)$ eigenvalue problems.
In particular, we aim to discuss two general criteria that guarantee the strong approximability condition that in the author's opinion are easier to verify than the strong approximability condition itself.
The first criterion is based on the inf-sup condition, which is the well-known necessary and sufficient condition for the well-posedness of the Stokes source problem and is also known to be a sufficient condition for the strong approximability condition, see \cite{BBF,BBG}.
The second criterion is the exactness and boundedness of the right inverse of the discrete curl operator. To the best of the author's knowledge, this criterion has never been discussed as a sufficient condition for the strong approximability condition for the Stokes eigenvalue problem.

To achieve this goal in section 2 we briefly recall the Boffi-Brezzi-Gastaldi (BBG) theory of the well-posedness of the mixed finite element method for $(f\!-\!0)$ eigenvalue problems and quickly review estimates for the order of convergence of the eigenvalues and eigenfunctions.
In section 3 we present our main results, i.e.~we investigate the scenario in which the inf-sup condition is not satisfied and we discuss how the strong approximability condition can be a consequence of the exactness and boundedness of the right inverse of the discrete curl operator or of the inf-sup condition on an auxiliary pressure space.
In section 4 we recast the result of section 3 in terms of finite element exterior calculus and comment on the role that the approximation of the divergence-free condition plays in the convergence of the discrete eigenvalue problem.
Lastly, in section 5 we present numerical results that confirm the theoretical findings of the previous sections.
The numerical examples presented include the use of $\mathbb{P}_1-\mathbb{P}_1$ elements on Powell-Sabin split meshes, $\mathbb{P}_2-\mathbb{P}_2$ elements on Alfeld split meshes and $\mathbb{P}_3-\mathbb{P}_2^{\,disc}$ elements on criss-cross split meshes for the Stokes eigenvalue problem in two dimensions.
We will also present in this section a new family of elements of the Bercovier--Pironneau type on Powll-Sabin and Alfeld splits and show that the rate of convergence for this family of elements fits the analysis presented in section 3.
The numerical example will also show that the rate of convergence of the eigenvalues is independent of any geometrical parameter of the mesh singularity, contrary to what is observed for the source problem.

\section{The necessary and sufficient for $(f\!-\!0)$ eigenvalue problems}
Since \cite{BBG} it has been long known that the well-posedness of finite element methods for mixed eigenvalue problems is different from the well-posedness of the source problem.
In fact, while the inf-sup condition is the necessary and sufficient condition for the well-posedness of the source problem, in the context of eigenvalue problems it is neither necessary nor sufficient.
In \cite{BBG} the authors showed that the inf-sup condition is a sufficient condition for the well-posedness of $(f-0)$ eigenvalue problems, but it is not necessary.
In this section, we will briefly recall the main results presented in \cite{BBG}.
To achieve this objective we will first introduce the bilinear forms:
\begin{align}
  a: \vec{V}\times \vec{V}\to \mathbb{R}, \qquad a(\vec{u},\vec{v})\coloneqq\int_{\Omega}\varepsilon(\vec{u}):\varepsilon(\vec{v})\, d\vec{x},\label{eq:a}\\
  b: \vec{V}\times Q\to \mathbb{R}, \qquad b(\vec{u},q)\coloneqq\int_{\Omega} (\nabla\cdot \vec{u})q\, d\vec{x},\label{eq:b}.
\end{align}
Using the bilinear forms defined above we can rewrite the eigenvalue problem \eqref{eq:stokes} as:
\begin{equation}
  \begin{aligned}
    2\mu\, a(\vec{u},\vec{v}) + b(\vec{v},p)&=\omega\, (\vec{u},\vec{v})_{L^2(\Omega)} \qquad &\forall \vec{v}\in \vec{V},\\
    -b(\vec{u},q)&=0 \qquad &\forall q\in Q.
  \end{aligned}
\end{equation}
We can also write in a compact form the source problem corresponding to \eqref{eq:stokes}, i.e.
\begin{equation}
  \label{eq:sourceStokes}
  \begin{aligned}
    2\mu\, a(\vec{u},\vec{v}) + b(\vec{v},p)&=(\vec{f},\vec{v})_{L^2(\Omega)} \qquad &\forall \vec{v}\in \vec{V},\\
    -b(\vec{u},q)&=0 \qquad &\forall q\in Q.
  \end{aligned}
\end{equation}
Since the \eqref{eq:sourceStokes} is well-posed at a continuous level, we can define the solution operator $S:\vec{V}\to \vec{V}\times Q$ as the operator that maps $\vec{f}\in \vec{V}$ to the solution of \eqref{eq:sourceStokes} when $\vec{f}$ is the right-hand side.
We will now focus our attention on the well-posedness of conforming mixed finite element methods for the eigenvalue problem \eqref{eq:stokes}, i.e. we fix two finite-dimensional subspaces $\vec{V}_h\subset \vec{V}$ and $Q_h\subset Q$ and we consider the following discrete eigenvalue problem:
\begin{equation}
  \label{eq:discreteStokes}
  \begin{aligned}
    2\mu\, a(\vec{u}_h,\vec{v}_h) + b(\vec{v}_h,p_h)&=\omega\, (\vec{u}_h,\vec{v}_h)_{L^2(\Omega)} \qquad &\forall \vec{v}_h\in \vec{V}_h,\\
    -b(\vec{u}_h,q_h)&=0 \qquad &\forall q_h\in Q_h.
  \end{aligned}
\end{equation}
and its corresponding source problem:
\begin{equation}
  \label{eq:discreteSourceStokes}
  \begin{aligned}
    2\mu\, a(\vec{u}_h,\vec{v}_h) + b(\vec{v}_h,p_h)&=(\vec{f}_h,\vec{v}_h)_{L^2(\Omega)} \qquad &\forall \vec{v}_h\in \vec{V}_h,\\
    -b(\vec{u}_h,q_h)&=0 \qquad &\forall q_h\in Q_h.
  \end{aligned}
\end{equation}
We would like to define a discrete counterpart of the solution operator $S$, yet as we know from \cite{Brezzi} such an operator does not exist unless the inf-sup condition is satisfied.
To overcome this issue, we introduce the following proposition:
\begin{proposition}[Proposition 6.5.2 \cite{BBF}]
  Since the bilinear form $a:\vec{V}_h\times \vec{V}_h\to \mathbb{R}$ is continuous and coercive on the kernel of the discrete divergence operator, i.e.~$\exists \alpha>0$ such that
  \begin{equation}
    a(\vec{v}_h,\vec{v}_h)\geq \alpha\, \norm{\vec{v}_h}^2_{\vec{V}} \qquad \forall \vec{v}_h\in \vec{K}_h\coloneqq \left\{\vec{v}_h\in \vec{V}_h\,|\, b(\vec{v}_h,q_h)=0\, \forall q_h\in Q_h\right\},
  \end{equation}
  then \eqref{eq:discreteSourceStokes} has at least one solution of the form $(\vec{u}_h,p_h)\in \vec{V}_h\times Q_h$. Furthermore, the $\vec{u}_h$ component of the solution is uniquely determined by the right-hand side $\vec{f}$ and the following estimate holds:
  \begin{equation}
    \norm{\vec{u}_h}_{\vec{V}}\leq \frac{1}{\alpha}\norm{\vec{f}_h}_{\vec{V}^\prime}.
  \end{equation}
  Hence, we can define the discrete solution operator $S_h:\vec{V}_h\to \vec{V}_h\times Q_h$ as the operator that maps $\vec{f}\in \vec{V}_h$ to a solution of \eqref{eq:discreteSourceStokes} when $\vec{f}$ is the right-hand side, suppose the one with the smallest norm.
\end{proposition}
Clearly, the existence of multiple discrete solutions when lacking the inf-sup condition forces us to consider a different approach to the well-posedness of the discrete eigenvalue problem, than just asking for $\norm{S-S_h}_{\mathcal{L}(\vec{V},\vec{V})}\to 0$ as $h\to 0$.
For this reason we we consider the following ``prolongation'' operators and the ``restriction'' operators defined as their adjoints:
\begin{equation}
  \begin{aligned}
    \pi_{\vec{V}}&:\vec{V}^\prime \to \vec{V}^\prime\times Q^\prime, \qquad &\vec{f}\mapsto (\vec{f},0),  \qquad\qquad \pi_{\vec{V}}^\ast&:\vec{V}\times Q \to \vec{V}, \qquad &(\vec{f},0) \mapsto \vec{f}\\ 
    \pi_{Q}&:Q^\prime \to \vec{V}^\prime\times Q^\prime, \qquad &g\mapsto (0,g),  \qquad\qquad \pi_{Q}^\ast&:\vec{V}\times Q \to Q, \qquad &(0,g) \mapsto g.
  \end{aligned}
\end{equation}
We now consider the restriction of the continuous and discrete solution operators defined as
\begin{equation}
  T\coloneqq \pi_{\vec{V}}^\ast\circ S\circ \pi_{\vec{V}}:\vec{H}\to \vec{H} \qquad \qquad \qquad T_h\coloneqq \pi_{\vec{V}}^\ast\circ S_h\circ \pi_{\vec{V}}:\vec{H} \to \vec{H}, 
\end{equation}
where $\vec{H}$ is the $\vec{L}^2(\Omega)$ space and we have identified $\vec{L}^2(\Omega)$ with its dual space and made used of the inclusion $\vec{V}\subset \vec{H}$ and $\vec{H}\subset \vec{V}^\prime$, hence we are not allowed any more Riesz identifications.
Notice now that from the previous proposition, we have that $T$ and $T_h$ are well-posed and uniquely defined, so the only thing left to do is to discuss the necessary and sufficient conditions so that $\norm{T-T_h}_{\mathcal{L}(\vec{H},\vec{H})}\to 0$ as $h\to 0$.
To achieve this goal we introduce two subspaces of $\vec{V}$ and $Q$ endowed with their natural norms, i.e.
\begin{equation}
  \begin{aligned}
    \vec{V}_0 &= \pi_{\vec{V}}^\ast\circ S\circ \pi_{\vec{V}}\left(\vec{L}^2(\Omega)\right), \qquad \norm{\vec{u}}_{\vec{V}_0} = \underset{\vec{f}\in\, T^{-1}\vec{v}}{\inf}\norm{\vec{f}}_{\vec{L}^2(\Omega)},\\
    Q_0 &= \pi_{Q}^\ast\circ S\circ \pi_{Q}\left(\vec{L}^2(\Omega)\right), \qquad \norm{q}_{\vec{Q}_0} = \underset{\vec{f}\in\, (\pi_Q^\ast \circ S \circ \pi_{\vec{V}})^{-1}\vec{v}}{\inf}\norm{\vec{f}}_{\vec{L}^2(\Omega)}.
  \end{aligned}
\end{equation}
\begin{definition}
  We say the pair $(Q_h,Q_0)$ satisfies the weak approximability condition if there exists a constant $C>0$ and an exponent $\gamma_1>0$ such that
  \begin{equation}
    \underset{\vec{v}_h\in \vec{K}_h}{\sup} \frac{b(\vec{v}_h,q)}{\norm{\vec{v}_h}_{\vec{V}}} \leq C h^{\gamma_1}\norm{q}_{\vec{Q}_0} \qquad \forall q\in Q_0.
  \end{equation}
\end{definition}
\begin{definition}
  We say the pair $(\vec{V}_h,\vec{V}_0)$ satisfies the strong approximability condition if there exists a constant $C>0$ and an exponent $\gamma_2>0$ such that
  \begin{equation}
    \label{eq:weakApprox}
    \underset{\vec{v}_h\in \vec{K}_h}{\inf} \norm{\vec{u}-\vec{v}_h}_{\vec{V}} \leq C h^{\gamma_2}\norm{\vec{u}}_{\vec{V}_0} \qquad \forall \vec{u}\in \vec{V}_0.
  \end{equation}
\end{definition}
\begin{theorem}[Theorem 6.5.1 and Theorem 6.5.2 \cite{BBF}]
  The weak approximability condition on $(Q_h,Q_0)$ and the strong approximability condition on $(\vec{V}_h,\vec{V}_0)$ are necessary and sufficient for the convergence of $T_h$ to $T$ in the operator norm $\norm{\cdot}_{\mathcal{L}(\vec{H},\vec{H})}$.
  In particular, if both conditions are satisfied, then the following estimate holds:
  \begin{equation}
    \norm{T\vec{f}-T_h\vec{f}}_{\vec{V}}\leq C h^{\min\{\gamma_1,\gamma_2\}}\norm{\vec{f}}_{\vec{H}}, \qquad \forall \vec{f}\in \vec{H}.
  \end{equation}
\end{theorem}
At this point, we have most of the ingredients we need to proceed with the analysis of the well-posedness and convergence of the discrete eigenvalue problem.
The only thing left to do is to discuss how to determine the order of convergence of the eigenvalues and eigenfunctions, once we have established the well-posedness of the discrete eigenvalue problem.
The most common approach to determine the order of convergence of the eigenvalues and eigenfunctions is to use the one presented in \cite{MercierOsbornRappazRaviart}.
Unfortunately, this approach can not used if the $S_h$ operator is not uniquely defined, as is the case when the inf-sup condition is not satisfied.
For this reason, we use the theory of approximation of compact operators, directly on $T_h:\vec{H}\to\vec{H}$ and $T:\vec{H}\to\vec{H}$, to determine the order of convergence of the eigenvalues and eigenfunctions, similarly to what is done in \cite{ErnGuermond, BrambleOsborn, Osborn, BabuskaOsborn}.
\begin{theorem}[Theorem 48.8 \cite{ErnGuermond}]
  Let $\omega$ be a non-zero eigenvalue of $T:\vec{H}\to\vec{H}$, with algebraic multiplicity $m_a(\omega)$. Notice that since $T:\vec{H}\to\vec{H}$ is self-adjoint, the algebraic multiplicity of the eigenvalues $\omega$ is equal to its geometric multiplicity $m_g(\omega)$.
  Let $\{\omega_h^{(j)}\}_{j=1}^{m_a(\omega)}$ be the eigenvalues of $T_h:\vec{H}\to\vec{H}$ converging to $\omega$ as $h\to 0$.
  Then, there exists a sufficiently small $h_0>0$ and a constant $C>0$ such that for all $h\leq h_0$ the following estimate holds:
  \begin{equation}
    \hat{\delta}(E_\omega,E_{h,\omega}) \leq C h^{2\min\{\gamma_1,\gamma_2\}},
  \end{equation}
  for all $h\leq h_0$, where $\hat{\delta}(E_\omega,E_{h,\omega})$ is the symmetric gap between the eigenspaces corresponding to $\omega$ and $\{\omega_h^{(j)}\}_{j=1}^{m_a(\omega)}$, respectively of $T$ and $T_h$.
  Furthermore, the following estimate holds for all $j\in \{1,\ldots,m_a(\omega)\}$:
  \begin{equation}
    \abs{\omega-\omega_h^{(j)}} \leq C h^{2\min\{\gamma_1,\gamma_2\}}.
  \end{equation} 
  Lastly, for every $j\in \{1,\ldots,m_a(\omega)\}$ and unit vector $\vec{w}_h^{(j)}\in ker\left(T_h-\omega_{h}^{(j)}I\right)$ there exists a unit vector $\vec{u}\in ker\left(T-\omega I\right)$ such that
  \begin{equation}
    \norm{\vec{u}-\vec{w}_h^{(j)}}_{\vec{H}}\leq C h^{\min\{\gamma_1,\gamma_2\}}.
  \end{equation}
\end{theorem}
\section{Sufficient conditions for the strong approximability condition}
In this section, we will discuss two general criteria that guarantee the strong approximability condition.
The reason why we focus our attention on the strong approximability condition and not on the weak approximability condition is that the weak approximability condition is a trivial consequence of the approximation properties of the finite element space $Q_h$.
In fact, following \cite{BBG} we observe that since $b(\vec{v}_h,q_h)$ vanish for all $\vec{v}_h\in \vec{K}_h$ and $q_h\in Q_h$ we subtract from \eqref{eq:weakApprox} to obtain
\begin{equation}
  \underset{\vec{v}_h\in \mathbb{K}_h}{sup} \frac{b(\vec{v}_h, q)}{\norm{\vec{v}_h}_{\vec{V}}}= \underset{\vec{v}_h\in \mathbb{K}_h}{sup} \frac{b(\vec{v}_h, q-q_h)}{\norm{\vec{v}_h}_{\vec{V}}}\leq \norm{q-q_h}_{Q}.
\end{equation}
The previous inequality shows that the weak approximability condition is verified as long as any function of $Q_0$ can be ``well'' approximated by a function in $Q_h$.
Focusing now on the strong approximability condition, the first criterion we will discuss is intimately related to the inf-sup condition and exactly divergence-free discretisation.
\begin{theorem}
  \label{thm:strongApproxInfSup}
  Let us consider a discrete eigenvalue problem of the type of \eqref{eq:discreteStokes} and assume the following approximation properties hold:
  \begin{align}
    \forall q\in Q, \quad \exists q_h\in Q_h \text{ such that } \norm{q-q_h}_{Q}\leq C_1 h^{\gamma_1}\norm{q}_{Q_0},\\
    \forall \vec{u}\in \vec{V}, \quad \exists \vec{v}_h\in \vec{V}_h \text{ such that } \norm{\vec{u}-\vec{v}_h}_{\vec{V}}\leq C_2 h^{\gamma_2}\norm{\vec{u}}_{\vec{V_0}},
  \end{align}
  for some $C_1,C_2>0$ and $\gamma_1,\gamma_2>0$.
  We denote the range of the discrete divergence operator as $\hat{Q}_h$, i.e.
  \begin{equation}
    \hat{Q}_h = \left\{q_h\in Q_h\,|\, \exists \vec{v}_h\in \vec{V}_h \text{ such that } \nabla\cdot \vec{v}_h=q_h\right\}.
  \end{equation}
  If the finite element pair $(\vec{V}_h,\hat{Q}_h)$ satisfies the inf-sup condition, with inf-sup constant $\beta>0$, then both the strong and weak approximability conditions are satisfied with $\gamma_1,\gamma_2$ by the the finite element pair $(\vec{V}_h,Q_h)$.
\end{theorem}
\begin{proof}
  We begin observing that if the inf-sup condition is satisfied by the pair $(\vec{V}_h,\hat{Q}_h)$, then the following estimate holds \cite[Proposition 5.1.3]{BBF}:
  \begin{equation}
    \underset{\vec{v}_h\in \hat{\mathbb{K}}_h}{\inf}\norm{\vec{v}-\vec{v}_h}_{\vec{V}}\leq \underset{\vec{v}_h\in \vec{V}_h}{\inf}\norm{\vec{v}-\vec{v}_h}_{\vec{V}}\leq C(\beta)h^{\gamma_2}\norm{\vec{v}}_{\vec{V}},\label{eq:strongApproxOptimalKernel}
  \end{equation}
  where $\hat{\mathbb{K}}_h\coloneqq \left\{\vec{v}_h\in \vec{V}_h\,|\, b(\vec{v}_h,q_h)=0\, \forall q_h\in \hat{Q}_h\right\}$.
  We notice that $\hat{\mathbb{K}}_h\subset \vec{K}_h$, in fact if we pick $\vec{v}_h\in \hat{\mathbb{K}}_h$ and $q_h = \nabla\cdot \vec{v}_h$ then $b(\vec{v}_h,q_h)=0$ implies that the divergence of $\vec{v}_h$ vanishes almost everywhere, hence $\vec{v}_h\in \mathbb{K}_h$.
  Therefore from \eqref{eq:strongApproxOptimalKernel} we have that the strong approximability condition is satisfied and the weak approximability condition is trivially satisfied as previously discussed.
\end{proof}
\begin{example}[Scott--Vogelius]
  One of the classical examples of divergence-free discretisation, in two dimensions, is the use of the Scott-Vogelius finite element pair $\left(\mathcal{P}^{k}(\mathcal{T}_h),\mathcal{P}^{k-1}_{disc}(\mathcal{T}_h)\right)$, where $\mathcal{T}_h$ is a simplicial mesh of the domain $\Omega$.
  While the Scott-.Vogelius pair seems the most natural choice for the Stokes eigenvalue problem, it is well known that the inf-sup condition is not satisfied for this pair for all values of $k\in\mathbb{B}$.
  In fact, it has been shown in \cite{GuzmanScott,ScottVogelius} that the inf-sup condition is satisfied for the Scott-Vogelius pair provided that $k\geq 4$ and we consider as pressure space the subspaces of $\mathcal{P}^{k-1}_{disc}(\mathcal{T}_h)$ defined as
  \begin{equation}
    \hat{Q}_h = \left\{q_h\in \mathcal{P}^{k-1}_{disc}(\mathcal{T}_h)\,|\, \sum_{j=1}^{N(\vec{z})} (-1)^{N-j} q_h\big|_{T_j(\vec{z})}(\vec{z}) = 0 \quad \forall \vec{z}\in \mathcal{S}(\mathcal{T}_h)\right\},
  \end{equation}
  where $\mathcal{S}(\mathcal{T}_h)$ is the set of all singular vertices of the mesh $\mathcal{T}_h$ and $T_j$ are the $N(\vec{z})$ simplices of the mesh $\mathcal{T}_h$ in the patch touching the singular vertex $\vec{z}$, i.e. vertices that are shared by two or more colinear edges. 
  Since the inf-sup condition is satisfied for $\left(\mathcal{P}^{k}(\mathcal{T}_h),\hat{Q}_h\right)$, we can apply Theorem \ref{thm:strongApproxInfSup} to conclude that the strong and weak approximability conditions are satisfied for the finite element pair $\left(\mathcal{P}^{k}(\mathcal{T}_h),\mathcal{P}^{k-1}_{disc}(\mathcal{T}_h)\right)$.
  This tells us that the Scott--Vogelius pair, with $k\geq 4$, is a good choice for the Stokes eigenvalue problem, independently of any geometrical restriction on the mesh.
  Under minor assumptions on the mesh it has been proven in \cite{GuzmanScott2} that there exists a pressure space $\hat{Q}_h$ such that the inf-sup condition is satisfied for the Scott-Vogelius pair with $k=3$, hence the strong approximability condition is satisfied for the finite element pair $\left(\mathcal{P}^{3}(\mathcal{T}_h),\mathcal{P}^{2}_{disc}(\mathcal{T}_h)\right)$.
  Therefore the Scott--Vogelius pair with $k=3$ is a good choice for the Stokes eigenvalue problem, provided that the mesh satisfies the minor assumptions of \cite{GuzmanScott2} as we show in Example \ref{ex:ScottVogelius3}.
\end{example}
\begin{example}[Alfeld split meshes]
  In \cite{ArnoldQin} it was shown that for two-dimensional domain the inf-sup condition is satisfied for the pair $\left(\mathbb{P}^2(\mathcal{T}_h),\mathbb{P}^1_{disc}(\mathcal{T}_h)\right)$ on special meshes known as Alfeld split meshes.
  Alfeld split meshes are a particular type of mesh where each triangle is split into three sub-triangles by connecting the vertices of the triangle with its barycenter.
  Interestingly, the Alfeld splits remove any singular vertex from the mesh, hence the pressure space $\hat{Q}_h$ is the full space $\mathbb{P}^1_{disc}(\mathcal{T}_h)$.
  Since the inf-sup condition is satisfied for $\left(\mathbb{P}^2(\mathcal{T}_h),\mathbb{P}^1_{disc}(\mathcal{T}_h)\right)$, we can apply Theorem \ref{thm:strongApproxInfSup} to conclude that the strong and weak approximability conditions are satisfied for the finite element pair $\left(\mathbb{P}^2(\mathcal{T}_h),\mathbb{P}^2_{disc}(\mathcal{T}_h)\right)$ and $\left(\mathbb{P}^2(\mathcal{T}_h),\mathbb{P}^1(\mathcal{T}_h)\right)$.
\end{example}
\bibliographystyle{plain} % We choose the "plain" reference style
\bibliography{metrix}

\end{document}
